% \iffalse meta-comment
%
% Copyright (C) 2021
% by Nanjing University Linux User Group <nju.lug@yaoge123.cn>
%
% It may be distributed and/or modified under the conditions of the
% LaTeX Project Public License (LPPL), either version 1.3c of this
% license or (at your option) any later version. The latest version
% of this license is in the file
%
%    https://www.latex-project.org/lppl.txt
%
% -----------------------------------------------------------------------
%
% The development version of the template can be found at
%
%    https://github.com/nju-lug/NJUVisual
%
% for those people who are interested.
%
% -----------------------------------------------------------------------
%
%<*internal>
\iffalse
%</internal>
%
%<*internal>
\fi
\begingroup
  \def\NameOfLaTeXe{LaTeX2e}
\expandafter\endgroup\ifx\NameOfLaTeXe\fmtname\else
\csname fi\endcsname
%</internal>
%
%<*install>
\input l3docstrip.tex
\keepsilent
\askforoverwritefalse

\preamble

Copyright (C) 2021
by Nanjing University Linux User Group <nju.lug@yaoge123.cn>

This file may be distributed and/or modified under the conditions of
the LaTeX Project Public License, either version 1.3 of this license
or (at your option) any later version.  The latest version of this
license is in:

   http://www.latex-project.org/lppl.txt

and version 1.3 or later is part of all distributions of LaTeX version
2005/12/01 or later.

To produce the documentation run the original source files ending with `.dtx'
through XeTeX.

\endpreamble

\generate{
  \usedir{tex/latex/njuvisual}
    \file{\jobname.sty} {\from{\jobname.dtx}{visual}
                        \from{\jobname-curves.dtx}{visual}}
%</install>
%<*internal>
  \usedir{source/latex/njuvisual}
    \file{\jobname.ins} {\from{\jobname.dtx}{install}}
%</internal>
%<*install>
}

\obeyspaces
\Msg{*************************************************************}
\Msg{*                                                           *}
\Msg{* To finish the installation you have to move the following *}
\Msg{* files into a directory searched by TeX:                   *}
\Msg{*                                                           *}
\Msg{* The recommended directory is TDS:tex/latex/njuvisual      *}
\Msg{*                                                           *}
\Msg{*     njuvisual.ins                                         *}
\Msg{*     njuvisual.sty                                         *}
\Msg{*                                                           *}
\Msg{* To produce the documentation, run the file njuvisual.dtx  *}
\Msg{* through XeLaTeX.                                          *}
\Msg{*                                                           *}
\Msg{* Happy TeXing!                                             *}
\Msg{*                                                           *}
\Msg{*************************************************************}

\endbatchfile
%</install>
%
%<*internal>
\fi
%</internal>
%
%<visual>\NeedsTeXFormat{LaTeX2e}
%<visual>\RequirePackage{expl3}
%<!driver>\GetIdInfo  $Id: njuvisual.dtx 0.1.0 2021-11-21 19:30:00 +0800  NJU LUG <nju.lug@yaoge123.cn> $
%<visual>  { LaTeX3 Package for NJU Visual Identity }
%<visual>\ProvidesExplPackage{njuvisual}
%<visual>{\ExplFileDate}{\ExplFileVersion}{\ExplFileDescription}
%
%<*driver>
\documentclass{ctxdoc}
\usepackage{multirow,njuvisual}
\setlist[1]{labelindent=0.5em}
\begin{document}
    \DocInput{njuvisual.dtx}
    \DocInput{njuvisual-curves.dtx}
    \PrintChanges
    \PrintIndex
\end{document}
%</driver>
%
% \fi
%
% \title{\color{njuviolet}{The \pkg{njuvisual} package\\ 南京大学视觉形象规范化标准}}
%
% \author{^^A
% Nanjing University Linux User Group
% \thanks{E-mail: \href{mailto:nju.lug@yaoge123.cn}{nju.lug@yaoge123.cn}}}
%
% \date{v0.1.0 \\ Released 2021-11-21}
%
% \changes{v0.1}{2021/11/21}{将\pkg{njuvisual}分离为单独宏包。}
%
% \maketitle
%
% \begin{abstract}
% 南京大学视觉形象规范化标准(\pkg{njuvisual})宏包负责实现\href{https://www.nju.edu.cn/3647/list.htm}{南京大学视觉形象规范化标准}以及各院系制定的配色方案和标识。
% \end{abstract}
%
% \def\abstractname{Abstract}
% \begin{abstract}
% The \pkg{njuvisual} package provides an interface to display logos related to Nanjing University.
% \end{abstract}
%
% \clearpage
%
% \tableofcontents
%
% \EnableDocumentation
%
%
% \begin{documentation}
%
% \section{宏包介绍}
%
% \pkg{njuvisual} 用于收录南京大学及其各院系制定的标准色彩及图标,并提供简洁的绘制指令。
%
% 本宏包衍生自 \pkg{njuthesis},依赖 \hologo{LaTeX3} 项目
% 中的 \pkg{expl3}、\pkg{xparse} 和 \pkg{l3keys2e} 宏包。
% 特别感谢 \pkg{fduthesis} 与 \pkg{zhlipsum} 提供的处理思路。
%
% \section{使用方法}
%
% \subsection{标准色彩}
%
% \begin{function}{\color}
%   \begin{syntax}
%     \tn{color}\Arg{颜色}
%   \end{syntax}
% 使用标准色彩。
% \end{function}
%
% 本宏包定义的所有标准色彩见表~\ref{tab:pre-defined-colors}。
% \begingroup
% \def\B#1{\color{#1}$\blacksquare\blacksquare\blacksquare\blacksquare\blacksquare$}
% \def\M#1{\multirow{2}*{#1}}
% \def\MN#1{\multirow{4}*{#1}}
% \begin{table}[htb]
%   \caption{预定义标准色} \label{tab:pre-defined-colors}
%   \centering\small
%   \begin{tabular}{cccc}
%     \toprule
%       名称                   & 颜色名称        & CMYK 数值                & 示例 \\
%     \midrule
%       \MN{南京大学}          & |njuviolet|     & $0.50, 1   , 0   , 0.40$ & \B{njuviolet} \\
%                              & |njumagenta|    & $0.05, 1   , 0.55, 0$    & \B{njumagenta} \\
%                              & |njublue|       & $0.80, 0.50, 0   , 0$    & \B{njublue} \\
%                              & |njuyellow|     & $0   , 0.30, 1   , 0$    & \B{njuyellow} \\
%       \M{人工智能学院}       & |nju-ai-blue|   & $0.75, 0.65, 0   , 0$    & \B{nju-ai-blue} \\
%                              & |nju-ai-orange| & $0   , 0.75, 1   , 0$    & \B{nju-ai-orange} \\
%       \M{计算机科学与技术系} & |nju-cs-blue|   & $0.89, 0.66, 0.13, 0$    & \B{nju-cs-blue} \\
%                              & |nju-cs-green|  & $0.60, 0.23, 1   , 0$    & \B{nju-cs-green} \\
%     \bottomrule
%   \end{tabular}
% \end{table}
% \endgroup
%
% \subsection{标准图标}
%
% \begin{function}[added=2021-09-24,updated=2021-11-27]{\njuemblem}
%   \begin{syntax}
%     \tn{njuemblem}\oarg{选项}\Arg{宽度}\Arg{高度}
%   \end{syntax}
%   生成徽标,默认生成校徽。参数 \meta{选项} 是可选的。注意各
%   参数之间不可以有空格。
% \end{function}
%
% \begin{function}[added=2021-09-24,updated=2021-11-27]{\njuname,\njuname*}
%   \begin{syntax}
%     \tn{njuname}\oarg{选项}\Arg{宽度}\Arg{高度}
%     \tn{njuname*}\oarg{选项}\Arg{宽度}\Arg{高度}
%   \end{syntax}
%   生成名称,带有可选星号的命令会生成英文名称,默认生成校名。要求同上。
% \end{function}
%
% 可选参数 \meta{选项} 可使用两种方式给出:
% \begin{itemize}
%   \item 颜色名称,使用颜色名称即默认生成校徽、校名
%   \item 使用英文半角逗号分隔的键值列表
% \end{itemize}
%
% 其中,键值列表支持的选项见下。
%
% \begin{function}[added=2021-11-27]{color,color*}
%   \begin{syntax}
%     color = \meta{主要颜色}
%     color* = \meta{辅助颜色}
%   \end{syntax}
%   定制图标颜色。
% \end{function}
%
% 对于绝大多数南京大学院系,其图标仅使用两种以内的颜色,因此本宏包仅提供两种颜色选项。
% 其中的 |color*| 选项一般无需修改,原因在于现有的图标可被总结为下列若干种情况:
% \begin{description}
%   \item[纯色、透明底色] 如南大徽标,无辅助颜色,|color*| 选项无效
%   \item[纯色、白色] 如现工徽标,辅助颜色默认为白色,无需进行修改
%   \item[双色] 如计科徽标,主要颜色、辅助颜色已预先进行定义,为美观考虑,修改主要颜色时会将图标颜色统一为一种
% \end{description}
%
% \begin{function}[added=2021-11-27]{department}
%   \begin{syntax}
%     department = \meta{院系名称}
%   \end{syntax}
%   院系名称。
% \end{function}
%
% 为保证无歧义,|department| 选项使用的名称来自于院系官网域名。如化学化工学院域名为\url{https://chem.nju.edu.cn/},其选项名称即为 |chem|。
%
% 目前的支持情况如表~\ref{tab:supported-departments-list}所示。对于不支持的院系,本宏包会抛出异常。
% \begingroup
% \def\B{\bullet}
% \begin{table}[htb]
%   \caption{收录院系列表} \label{tab:supported-departments-list}
%   \centering\small
%   \begin{tabular}{cccccccc}
%     \toprule
%       名称 & 选项名称 & emblem & name & name* \\
%     \midrule
%       匡亚明学院             & |dii| & \B & & \\
%       现代工程与应用科学学院 & |eng| & \B & & \\
%       人工智能学院           & |ai| & \B & & \\
%       计算机科学与技术系     & |cs| & \B & & \\
%     \bottomrule
%   \end{tabular}
% \end{table}
% \endgroup
%
% 举例如下:
% \begin{ctexexam}
%   \njuemblem{!}{3cm}                           % 默认生成指定大小的紫色南大校徽
%   \njuname{4cm}{!}                             % 默认生成指定大小的紫色南大中文校名
%   \njuname*{4cm}{!}                            % 默认生成指定大小的紫色南大英文校名
%   \njuemblem[black]{!}{3cm}                    % 黑色的南大校徽
%   \njuemblem[department=dii]{!}{4cm}           % 紫色匡院徽标
%   \njuemblem[department=cs,color=blue]{!}{3cm} % 纯蓝色计科徽标
% \end{ctexexam}
%
%
% \begin{function}[added=2021-09-24]{\njumotto}
%   \begin{syntax}
%     \tn{njumotto}\oarg{颜色}\Arg{宽度}\Arg{高度}
%   \end{syntax}
% 生成指定颜色和大小的南京大学校训。
% \end{function}
%
%
% \begin{function}[added=2021-09-24]{\njuspirit}
%   \begin{syntax}
%     \tn{njuspirit}\oarg{颜色}\Arg{宽度}\Arg{高度}
%   \end{syntax}
% 生成指定颜色和大小的南京大学校风。
% \end{function}
%
% \end{documentation}
%
% \EnableImplementation
%
% \begin{implementation}
%
% \section{实现细节}
%
%    \begin{macrocode}
%<@@=njuvis>
%<*visual>
%    \end{macrocode}
%
% \subsection{载入宏包}
%
% 导入 \pkg{tikz} 宏包,其中已包含 \pkg{xcolor},无需单独加载。
%    \begin{macrocode}
\RequirePackage { tikz }
%    \end{macrocode}
%
% \subsection{定义变量}
%
% \begin{macro}{\l_@@_color_a_tl,\l_@@_color_b_tl}
% 存储使用的颜色。
%    \begin{macrocode}
\tl_new:N \l_@@_color_a_tl
\tl_new:N \l_@@_color_b_tl
%    \end{macrocode}
% \end{macro}
%
% \begin{macro}{\l_@@_department_tl}
% 存储院系
%    \begin{macrocode}
\tl_new:N \l_@@_department_tl
%    \end{macrocode}
% \end{macro}
%
% \changes{v0.1}{2021/11/27}{使用键值对处理院系和颜色。}
%    \begin{macrocode}
\keys_define:nn { njuvisual }
  {
%    \end{macrocode}
% \begin{macro}{color,color*}
% 颜色。设定初始辅助色为白色
%    \begin{macrocode}
    color       .tl_set:N = \l_@@_color_a_tl,
    color      .initial:n = default,
    color*      .tl_set:N = \l_@@_color_b_tl,
    color*     .initial:n = white,
%    \end{macrocode}
% \end{macro}
%
% \begin{macro}{department}
% 院系
%    \begin{macrocode}
    department  .tl_set:N = \l_@@_department_tl,
    department .initial:n = nju,
  }
%    \end{macrocode}
% \end{macro}
%
% \begin{variable}{\c_@@_department_all_clist}
% 定义南京大学全体院系名单。
%    \begin{macrocode}
\clist_const:Nn \c_@@_department_all_clist
{ nju, dii, eng, chin, history, philo, jc, law, nubs, sfs, im, sociology, 
  math, physics, astronomy, chem, cs, software, ai, ese, hjxy, sgos,
  as, life, med, sme, hwxy, arch, marxism, art, edu }
%    \end{macrocode}
% \end{variable}
%
% \begin{variable}{\c_@@_department_emblem_clist}
% 定义受图标支持的南京大学院系名单。
%    \begin{macrocode}
\clist_const:Nn \c_@@_department_emblem_clist { nju, dii, eng, cs, ai }
%    \end{macrocode}
% \end{variable}
%
% \begin{variable}{\c_@@_department_name_clist}
% 定义受名称支持的南京大学院系名单。
%    \begin{macrocode}
\clist_const:Nn \c_@@_department_name_clist   { nju }
%    \end{macrocode}
% \end{variable}
%
% \begin{variable}{\c_@@_department_color_clist}
% 定义受色彩支持的南京大学院系名单。
%    \begin{macrocode}
\clist_const:Nn \c_@@_department_color_clist  { cs, ai }
%    \end{macrocode}
% \end{variable}
%
% 定义对不支持的院系的报错信息。
%    \begin{macrocode}
\msg_new:nnn { njuvisual } { unsupported-department }
{ "#1"~ is~ not~ an~ valid~ option~ \\\\
  Please~ refer~ to~ the~ package~ documentation~  for~ supported~ 
  departments~ and~ check~ your~ spelling. \\ }
%    \end{macrocode}
%
% \subsection{内部命令}
%
% \begin{macro}{\@@_set_color:}
% \changes{v0.1}{2021/11/25}{完善设定颜色的方法。}
% 设定颜色。
%    \begin{macrocode}
\cs_new_protected:Nn \@@_set_color:
  {
    \str_if_eq:VnTF \l_@@_color_a_tl { default }
      { 
%    \end{macrocode}
% 如果没有单独设定,则使用默认颜色。
%    \begin{macrocode}
        \clist_if_in:NVTF \c_@@_department_color_clist \l_@@_department_tl
          { \tl_set:Nn \l_@@_color_a_tl 
              { \clist_item:cn { c_@@_color_ \l_@@_department_tl _clist } { 1 } }
            \tl_set:Nn \l_@@_color_b_tl 
              { \clist_item:cn { c_@@_color_ \l_@@_department_tl _clist } { 2 } } }
          { \tl_set:Nn \l_@@_color_a_tl { njuviolet } }
      } 
      { \clist_if_in:NVT \c_@@_department_color_clist \l_@@_department_tl
%    \end{macrocode}
% 将双色图标设定为纯色。
%    \begin{macrocode}
          { \tl_set_eq:NN \l_@@_color_b_tl \l_@@_color_a_tl } }
  }
%    \end{macrocode}
% \end{macro}
%
% \begin{macro}{\@@_tikz_wrapper:nnn}
% 可变大小的 tikz 容器。\footnote{此处手动调整了列表缩进,实际上\env{arguments}环境的缩进问题已在\href{https://github.com/CTeX-org/ctex-kit/commit/838ebab8378b6b0477f5aece81ace28ae2a8bcd9}{838ebab}提交中修复}
% \begin{arguments}
%   \item 路径
%   \item 宽度
%   \item 高度
% \end{arguments}
% 封装 resizebox 和 tikzpicture 环境,减少重复。
%    \begin{macrocode}
\cs_new_protected:Npn \@@_tikz_wrapper:nnn #1#2#3
{
  \resizebox { #2 } { #3 } {
    \begin{tikzpicture}[y=0.80pt, x=0.80pt, yscale=-1,
        xscale=1, inner~sep=0pt, outer~sep=0pt] #1
    \end{tikzpicture} }
}
%    \end{macrocode}
% \end{macro}
%
% \begin{macro}{\@@_if_key_value_list:nTF}
% 判断是否为键值列表,即是否含有 |=|。
%    \begin{macrocode}
\cs_new_protected:Npn \@@_if_key_value_list:nTF #1
  { \tl_if_in:nnTF {#1} {=} }
%    \end{macrocode}
% \end{macro}
%
% \begin{macro}{\@@_colon}
% 定义类别为 other 的冒号,用于在 Expl3 中表示路径
%    \begin{macrocode}
\def \@@_colon { \char_generate:nn { 58 } { 12 } }
%    \end{macrocode}
% \end{macro}
%
%
% \subsection{用户接口}
%
% \begin{macro}{\njuemblem}
% \changes{v0.1}{2021/11/24}{将学校和院系的徽标和名称命令统一。}
% 定义南京大学标准徽标。
% \begin{arguments}
%   \item 颜色或者键值对
%   \item 宽度
%   \item 高度
% \end{arguments}
%    \begin{macrocode}
\NewDocumentCommand \njuemblem { O { njuviolet } m m }
{
  \group_begin:
%    \end{macrocode}
% 根据参数是否含有 |=| 来判断该参数是段落数还是选项列表。
%    \begin{macrocode}
    \@@_if_key_value_list:nTF { #1 }
    {
      \keys_set:nn { njuvisual } { #1 }
%    \end{macrocode}
% 检查是否受支持。
%    \begin{macrocode}
          \clist_if_in:NVTF \c_@@_department_emblem_clist \l_@@_department_tl
        { \@@_set_color:
          \@@_tikz_wrapper:nnn 
            { \use:c { @@_path_emblem_ \l_@@_department_tl } } { #2 } { #3 } }
        { \msg_error:nnx { njuvisual } 
            { unsupported-department } { \l_@@_department_tl } }
    }
    { \tl_set:Nn \l_@@_color_a_tl { #1 }
      \@@_tikz_wrapper:nnn { \@@_path_emblem_nju } { #2 } { #3 } }
  \group_end:
}
%    \end{macrocode}
% \end{macro}
%
%
% \begin{macro}{\njuname,\njuname*}
% 定义南京大学标准名称。
% \begin{arguments}
%   \item 可选星号
%   \item 颜色或者键值对
%   \item 宽度
%   \item 高度
% \end{arguments}
%    \begin{macrocode}
\NewDocumentCommand \njuname { t* O { njuviolet } m m }
{
  \group_begin:
    \@@_if_key_value_list:nTF { #2 }
    {
      \keys_set:nn { njuvisual } { #2 }
      \clist_if_in:NVTF \c_@@_department_name_clist \l_@@_department_tl
        { 
          \@@_set_color:
          \bool_if:NTF { #1 }
            { \@@_tikz_wrapper:nnn 
                { \use:c { @@_path_name_ #2 _en } } { #3 } { #4 } }
            { \@@_tikz_wrapper:nnn 
                { \use:c { @@_path_name_ #2 _cn } } { #3 } { #4 } }
        }
        { \msg_error:nnx { njuvisual } 
            { unsupported-department } { \l_@@_department_tl } }
    }
    {
      \tl_set:Nn \l_@@_color_a_tl { #2 }
      \bool_if:NTF { #1 }
        { \@@_tikz_wrapper:nnn { \@@_path_name_nju_en } { #3 } { #4 } }
        { \@@_tikz_wrapper:nnn { \@@_path_name_nju_cn } { #3 } { #4 } }
    }
  \group_end:
}
%    \end{macrocode}
% \end{macro}
%
%
% \begin{macro}{\njumotto}
% 定义南京大学校训。
% \begin{arguments}
%   \item 颜色
%   \item 宽度
%   \item 高度
% \end{arguments}
%    \begin{macrocode}
\NewDocumentCommand \njumotto  { o m m }
{ 
  \group_begin:
    \tl_set:Nn \l_@@_color_a_tl { #1 }
    \@@_tikz_wrapper:nnn { \@@_path_motto } { #2 } { #3 }
  \group_end:
}
%    \end{macrocode}
% \end{macro}
%
%
% \begin{macro}{\njuspirit}
% 定义南京大学校风。
% \begin{arguments}
%   \item 颜色
%   \item 宽度
%   \item 高度
% \end{arguments}
%    \begin{macrocode}
\NewDocumentCommand \njuspirit { o m m }
{ 
  \group_begin:
    \tl_set:Nn \l_@@_color_a_tl { #1 }
    \@@_tikz_wrapper:nnn { \@@_path_spirit } { #2 } { #3 } 
  \group_end:
}
%</visual>
%    \end{macrocode}
% \end{macro}
%
